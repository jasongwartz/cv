%%%%%%%%%%%%%%%%%%%%%%%%%%%%%%%%%%%%%%%%%
% Medium Length Professional CV
% LaTeX Template
% Version 2.0 (8/5/13)
%
% This template has been downloaded from:
% http://www.LaTeXTemplates.com
%
% Original author:
% Trey Hunner (http://www.treyhunner.com/)
%
% Important note:
% This template requires the resume.cls file to be in the same directory as the
% .tex file. The resume.cls file provides the resume style used for structuring the
% document.
%
%%%%%%%%%%%%%%%%%%%%%%%%%%%%%%%%%%%%%%%%%

%----------------------------------------------------------------------------------------
%	PACKAGES AND OTHER DOCUMENT CONFIGURATIONS
%----------------------------------------------------------------------------------------

\documentclass{resume} % Use the custom resume.cls style

\usepackage[left=0.75in,top=0.6in,right=0.75in,bottom=0.6in]{geometry} % Document margins
% \usepackage[lowtilde]{url}
\usepackage[hidelinks]{hyperref}
\urlstyle{same}

\name{Jason Gwartz}
\address{+44 7564 675581 ~$\cdot$~ jason@gwartz.me}
\address{\url{https://github.com/jasongwartz} ~$\cdot$~ \url{https://www.linkedin.com/in/jason-gwartz/}}

\begin{document}

\vspace{5 mm}
%----------------------------------------------------------------------------------------
%	WORK EXPERIENCE SECTION
%----------------------------------------------------------------------------------------

\begin{rSection}{Experience}

\begin{rSubsection}{{Apple}}{2020-Present}{Senior Site Reliability Engineer and Tech Lead}{London}
    \item Collaborated with other leaders to plan the roadmap for a "serverless" functions-as-a-service product
    \item Designed a configuration-as-code system to speed up rollouts from days to minutes
    \item Planned and executed a multi-year transition to Kubernetes from legacy platforms
    \item Contributed to Terraform and Pulumi infrastructure-as-code systems to deploy AWS resources
    \item Led a transition from "toil" to an engineering-oriented SRE culture, improving scalability
    \item Developed software in Go, TypeScript, and server-side Swift
\end{rSubsection}

\begin{rSubsection}{{Ecosia.org}}{2018-2020}{Software Developer and Tech Lead}{Berlin}
    \item Maintained a Kubernetes-based production infrastructure on AWS, having led and completed a full migration to Docker and Kubernetes
    \item Led technical projects (eg. production observability, infrastructure migration, A/B testing infrastructure, user accounts service)
    \item Worked collaboratively with a Product Owner to estimate and plan development and set goals
    \item Owned and supported a CI/CD system and developer build tooling
    \item Participated in an on-call rotation and trained developers in incident response
    \item Contributed to open-source projects (eg. Kubernetes community tools)
    \item Developed software in Go, Python, and JavaScript (Node.js and Vue/Nuxt)
    \item Additional technologies: Make, Bazel, Docker, Prometheus, nginx, PostgreSQL, Terraform, Vault
\end{rSubsection}

%------------------------------------------------

\begin{rSubsection}{{Mycujoo}}{2017-2018}{Devops Engineer}{Amsterdam}
    \item Managed multiple Kubernetes clusters in cloud infrastructure
    \item Developed applications and tooling for Kubernetes in Node.js and Go
    \item Managed monitoring and alerting tools (Elasticsearch/ELK, Prometheus, Grafana)
\end{rSubsection}

%------------------------------------------------

\begin{rSubsection}{{Booking.com}}{2016-2017}{Graduate Software Developer}{Amsterdam}
    \item Deployed a real-time machine learning prediction system using Docker and Openshift (Kubernetes)
    \item Worked collaboratively with UX designers and front-end developers to implement and A/B test customer-facing features
    \item Investigated and monitored application vulnerabilities and responded to security incidents
    \item Additional technologies: Perl, nginx, Python, A/B testing, Docker, Kubernetes, Openshift, Tensorflow
\end{rSubsection}

\begin{rSubsection}{{Hult International Business School}}{2015-2016}{Campus Technology Engineer}{London}
    \item Developed and maintained automation utilities to extend the functionality of the Canvas LMS and Basecamp project management platforms using their REST APIs
\end{rSubsection}

\end{rSection}

\newpage

%----------------------------------------------------------------------------------------
%	VOLUNTEER EXPERIENCE SECTION
%----------------------------------------------------------------------------------------

\begin{rSection}{Volunteer}

    \begin{rSubsection}{{Manara}}{2019 - 2020}{Volunteer Coding Teacher}{Remote}
        \item Mentor and coach young Palestinian developers on getting their first software development job
        \item Conduct training classes on computer science fundamentals
        \item Provide one-on-one support and career guidance
    \end{rSubsection}

    %------------------------------------------------

    \begin{rSubsection}{{HackYourFuture}}{2017 - 2018}{Volunteer Coding Teacher for Refugees}{Amsterdam}
        \item Help provide a free software development education to refugees in the Netherlands
        \item Teach weekly classes on Node.js backend development and relational databases (eg. MySQL)
        \item Design the curriculum for the Node.js and Databases courses
        \item Mentor students in independent projects
        \item Provide students with coding interview support and practice
    \end{rSubsection}

    \begin{rSubsection}{{Various (Other)}}{}{Volunteer Teacher or Mentor}{}
        \item Occasional teacher or mentor with organizations such as Pyladies and Frauenloop
    \end{rSubsection}

    %------------------------------------------------

\end{rSection}

%----------------------------------------------------------------------------------------
%	EDUCATION SECTION
%----------------------------------------------------------------------------------------

\begin{rSection}{Education}
    
    \begin{rSubsection}{{University College London}}{2015-2016}{Master of Computer Science {\em (with Distinction)}}{London, UK} 
        \item Dissertation: "Chopsticks! A web-based graphical programming language for live-coding music"
        \item Awarded the Peter Williams Prize for Best Project in MSc Computer Science
    \end{rSubsection}
    
    
    \begin{rSubsection}{{York University}}{2009-2013}{Bachelor of Fine Arts in Music {\em (summa cum laude)}}{Toronto, Canada}
        \item[]
    \end{rSubsection}
    
\end{rSection}
    
%----------------------------------------------------------------------------------------
%	TECHNICAL STRENGTHS SECTION
%----------------------------------------------------------------------------------------

% \begin{rSection}{Technical Strengths}

% \begin{tabular}{ @{} >{\bfseries}l @{\hspace{6ex}} l }
% Computer Languages & Prolog, Haskell, AWK, Erlang, Scheme, ML \\
% Protocols \& APIs & XML, JSON, SOAP, REST \\
% Databases & MySQL, PostgreSQL, Microsoft SQL \\
% Tools & SVN, Vim, Emacs
% \end{tabular}

% \end{rSection}

%----------------------------------------------------------------------------------------
%	EXAMPLE SECTION
%----------------------------------------------------------------------------------------

%\begin{rSection}{Section Name}

%Section content\ldots

%\end{rSection}

%----------------------------------------------------------------------------------------

\end{document}
